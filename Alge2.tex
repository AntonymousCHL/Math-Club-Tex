\documentclass{article}

\usepackage[utf8]{inputenc}
\usepackage[letterpaper, total={6in, 9in}]{geometry}
\usepackage{amsmath}
\usepackage{natbib}
\usepackage{wrapfig}
\usepackage{graphicx}
\usepackage{amssymb}
\usepackage{tikz}
\usepackage{enumitem}


\title{Algebra 2 - Functional Equations}
\author{TSS Math Club}
\date{Jan 2023}

\begin{document}
\large

\maketitle

\section{Introduction:}

\subsection{Basic Types of Functions:}
\begin{itemize}
    \item Linear
    \item Quadratic
    \item Square Root
    \item Reciprocal
\end{itemize}

\subsection{Equations:}
Statement that shows two (or more) mathematical expressions are equal.
\\For example, solve for $x$: $2x = 10 - \frac{x}{4}$
\\~\\ In a functional equation, the unknown is not a value, it is a function.
To solve a functional equation, find the relationship between the input $x$ and the output $f(x)$
\subsection{Functional Equations:}
There are usually two conditions given:
\begin{itemize}
    \item Equation
    \item Domain and Range
\end{itemize}
Any solution given must satisfy those two conditions.
\vspace{40px}

\section{Basic Examples}
\subsection{Solving Functional Equations}
\begin{itemize}
    \item Substitution
    \item Induction
\end{itemize}
\subsection{Example}
$$
f(x+3)=x^2+5x
$$
\\Determine $f(x)$
\vspace{240px}

\subsection{Example}
$$
f(\frac{2x-1}{x-3})=x^2
$$
Determine $f(x)$
\pagebreak

\section{Problems}
\subsection{Cauchy's Functional Equation}
Find all functions $f:\mathbf{Q}\rightarrow\mathbf{Q}$ such that
$$
f(x+y)=f(x)+f(y)
$$
for all $x, y \in \mathbf{Q}$
\vspace{220px}

\subsection{Problem}
$$
f(x-y)=f(x)+f(y)-2xy
$$
\pagebreak

\subsection{2020 CSMC A6}
Suppose that $f(x)$ is a function defined for every real number $x$ with $0 \le x \le 1$ with the properties that:
\begin{itemize}
    \item $f(1-x)=1-f(x)$ for all real numbers $x$ with $0 \le x \le 1$,
    \item $f(\frac{1}{3}x=\frac{1}{2}f(x)$ for all real numbers $x$ with $0 \le x \le 1$, and
    \item $f(a) \le f(b)$ for all real numbers $0 \le a \le b \le 1$.
\end{itemize}
What is the value of $f(\frac{6}{7})$?
\pagebreak

\subsection{2021 CSMC B3}
A pair of functions $f(x)$ and $g(x)$ is called a \emph{Payneful pair} if:
\begin{enumerate}[label=(\roman*)]
    \item $f(x)$ is a real number for all real numbers $x$,
    \item $g(x)$ is a real number for all real numbers $x$,
    \item $f(x + y) = f(x)g(y) + g(x)f(y)$ for all real numbers $x$ and $y$,
    \item $g(x + y) = g(x)g(y) - f(x)f(y)$ for all real numbers $x$ and $y$, and
    \item $f(a) \neq 0$ for some real number a.
\end{enumerate}
\vspace{20px}
For every \emph{Payneful pair} of functions $f(x)$ and $g(x)$:
\begin{enumerate}[label=(\alph*)]
    \item Determine the values of $f(0)$ and $g(0)$.
    \item If $h(x) = (f(x))^2 + (g(x))^2$, for all real numbers $x$, determine the value of $h(5)h(-5)$.
    \item If $-10 \le f(x) \le 10$ and $-10 \le g(x) \le 10$, for all real numbers $x$, determine the value of $h(2021)$.
\end{enumerate}
\pagebreak

\end{document}