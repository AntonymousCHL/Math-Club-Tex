\documentclass{article}

\usepackage[utf8]{inputenc}
\usepackage[letterpaper, total={6in, 9in}]{geometry}
\usepackage{amsmath}
\usepackage{natbib}
\usepackage{wrapfig}
\usepackage{graphicx}
\usepackage{amssymb}
\usepackage{tikz}


\title{Algebra 2 - Sequence and Series}
\author{TSS Math Club}
\date{Jan-Feb 2023}

\begin{document}
\large

\maketitle

\section{Sequence}

\subsection{Definition}
A sequence can be thought of as a list of numbers written in a definite order:
$$a_1, a_2, a_3, a_4,\dots , a_n, \dots$$
The number $a_1$ is called the first term, $a_2$ is the second term, and in general $a_n$ is the n-th
term.

\subsection{Notation}
The sequence $\{a_1, a_2, a_3, \dots\}$ is also denoted by $\{a_n\}$ or $\{a_n\}_{n \ge 1}$


\subsection{Examples}

\begin{itemize}
    \item $a_n = \frac{1}{n}$
    \item $a_1 = 1, a_{n+1}=a_{n}+2$
    \item $a_1=1, a_2=1, a_{n+2}=a_{n+1}+a_{n} $
\end{itemize}


\subsection{Arithmetic Sequence}

\subsubsection{Definition}

An arithmetic sequence is a sequence where each term increases by adding/subtracting some constant $d$ (the common difference).

$$a_{n+1}=a_{n}+d$$

\subsubsection{Useful Formulae}

\begin{itemize}
    \item $a_{n}=a_1+(n-1)d$
    \item $2a_{n}=a_{n-1}+a_{n+1}$
\end{itemize}

\subsection{Geometric Sequence}

\subsubsection{Definition}

A geometric sequence is a sequence in which each term is found by multiplying the preceding term by the same value $r$ (common ratio).

$$a_{n+1}=ra_{n}$$

\subsubsection{Useful Formulae}

\begin{itemize}
    \item $a_{n}=a_1r^{n-1}$
    \item $a_{n}^2=a_{n-1}a_{n+1}$
\end{itemize}

\subsection{Arithmetico-Geometric Sequence}

\subsubsection{Definition}

In mathematics, arithmetico-geometric sequence is the result of term-by-term multiplication of a geometric progression with the corresponding terms of an arithmetic progression. 

$$a_n=[a+(n-1)d]br^{n-1}$$

\subsection{Recursive Sequence}
A recursive sequence is a sequence in which terms are defined using one or more previous terms which are given.

\subsubsection{Periodic Sequence}

Example : $a_1=x,a_2=y, a_n=\frac{a_{n-1}+1}{a_{n-2}}$
\\

$a_1=x,a_2=y,a_3=\frac{y+1}{x}, a_4=\frac{x+y+1}{xy}, a_5 = \frac{x+1}{y}, a_6=x, a_7=y, \dots$

\subsubsection{Linear Difference Equation}

Example : $a_1=1, a_2=1, a_{n+2}=a_{n+1}+a_{n} $
\\
How to solve:
\begin{itemize}
    \item Step 1: find the characteristic equation $x^2=x+1$
    \item Step 2: Solve the characteristic equation $x_{1,2}=\phi,\frac{1}{\phi}$
    \item Step 3: the general term $a_n = c_1 x_1^n+ c_2 a_2^n$ for some constant $c_i$
    \item Step 4: sub $a_1$ and $a_2$ in and get $c_i$
\end{itemize}

\section{Series}

\subsection{Defnition}

A series is the cumulative sum of a given sequence of terms.

$$S_n=\sum_{i=1}^{n}a_i \mathrm{,for \ some} \{a_i\}$$ 

\subsection{Arithmetic Series}

Let $\{a_n\}_{n\ge1}$ be a arithmetic sequence, then 

$$S_n=\sum_{i=1}^{n}a_i$$ 
is the arithmetic series.

\subsubsection{General Term}

$a_n = \frac{(a_1+a_n)}{2}$

\subsubsection{Prove for $a_n=n$}

consider:
\\
1,2,3,4,5,...,n-2,n-1,n\\
n,n-1,n-2,...,5,4,3,2,1

If we add vertically, we'll get n+1 for n terms.
But because this is for $2S_n$, $$S_n=\frac{(n+1)n}{2}$$

\subsubsection{Property of sum operator}

IT IS A LINEAR OPPERATOR:

\begin{itemize}
    \item $\sum (a_i+b_i) = \sum a_i + \sum b_i$
    \item $\sum ca_i = c \sum a_i$ for constant $c$
\end{itemize}

\subsubsection{Proof of genral term}
\newpage

\subsection{Geometric Series}

Let $\{a_n\}_{n\ge1}$ be a geometric sequence, then 

$$S_n=\sum_{i=1}^{n}a_i$$ 
is the geometric series.

\subsubsection{General Term}

$a_n = \frac{a_1(r^{n+1}-1)}{r-1}$

\subsubsection{Prove for $a_n=r^{n-1}$}
By definition, we have
$$S_n=1+r+r^2+\dots+r^{n-1}$$
Now, consider the identity $$x^n-1=(x-1)(x^{n-1}+\dots+x^2+x+1)$$
Therefore,
$$S_n=1+r+r^2+\dots+r^{n-1}=\frac{r^n-1}{r-1}$$

\subsubsection{Proof of genral term}
\vspace{30px}
\subsubsection{Push to infinity}

Note if $-1<r<1$, we have $\lim_{n \to \infty} r^n=0$
Therefore, we have $$\lim_{n \to \infty} S_n = \lim_{n \to \infty} \frac{a_1(r^{n+1}-1)}{r-1} = \frac{a_1(\lim_{n \to \infty} r^{n+1}-1)}{r-1} = \frac{-a_1}{r-1} = \frac{a_1}{1-r}$$

\subsection{Arithmetico-Geometric Series}

Let $\{a_n\}_{n\ge1}$ be a arithmetico-geometric sequence, then 

$$S_n=\sum_{i=1}^{n}a_i$$ 
is the arithmetico-geometric series.

\subsection{General term}
$x_n=(a_1+d(n-1))(g_1\cdot r^{n-1})$\\
Let $S_n$ represent the sum of the first $n$ terms. \\
$S_n=a_1g_1+(a_1+d)(g_1r)+(a_1+2d)(g_1r^2)+\ldots+(a_1+(n-1)d)(g_1r^{n-1})$\\
$S_n=a_1g_1+(a_1g_1+dg_1)r+(a_1g_1+2dg_1)r^2+\ldots+(a_1g_1+(n-1)dg_1)r^{n-1}$\\
$rS_n=a_1g_1r+(a_1g_1+dg_1)r^2+(a_1g_1+2dg_1)r^3+\ldots+(a_1g_1+(n-1)dg_1)r^{n}$\\
$rS_n-S_n=-a_1g_1-dg_1r-dg_1r^2-dg_1r^3-\ldots-dg_1r^{n-1}+(a_1g_1+(n-1)dg_1)r^n$\\
$S_n(r-1)=(a_1+(n-1)d)g_1r^n-a_1g_1-\frac{dg_1r(r^{n-1}-1)}{r-1}$\\
$S_n=\frac{(a_1+(n-1)d)g_1r^n}{r-1}-\frac{a_1g_1}{r-1}-\frac{dg_1r(r^{n-1}-1)}{(r-1)^2}=\frac{a_ng_{n+1}}{r-1}-\frac{x_1}{r-1}-\frac{d(g_{n+1}-g_2)}{(r-1)^2}$

\subsection{Telescoping Series}

\subsubsection{Example}

What's the value of $$\frac{1}{1\cdot2}+\frac{1}{2\cdot3}+\dots+\frac{1}{(n-1)\cdot n}$$
\vspace{100px}

\section{Practice Problems}

\subsection{2021 CIMC Part A Q5}

A list of numbers is created using the following rules:
\begin{itemize}
    \item The first number is 3 and the second number is 4.
    \item Each number after the second is the result of adding 1 to the previous number
    and then dividing by the number before that. In other words, for any three
    consecutive numbers in the list, $a, b, c,$ we have $\frac{b+1}{a}$
\end{itemize}
What is the smallest positive integer $N$ for which the sum of the first $N$ numbers in
the list is equal to an odd integer that is greater than 2021?
\newpage
\subsection{2022 Euclid Q5}
A list $a_1, a_2, a_3, a_4$ of rational numbers is defined so that if one term is equal to $r$,
then  the next term is equal to $1+\frac{1}{1+r}$.  If $a_3=\frac{41}{29}$, what is the value of a1?
\vspace{100px}

\subsection{2005 AIME II Q3}

An infinite geometric series has sum 2005. A new series, obtained by squaring each term of the original series, has 10 times the sum of the original series. The common ratio of the original series is $\frac mn$ where $m$ and $n$ are relatively prime integers. Find $m+n.$

\vspace{100px}

\subsubsection{2022 AMC 12A Q8}
The infinite product\[\sqrt[3]{10} \cdot \sqrt[3]{\sqrt[3]{10}} \cdot \sqrt[3]{\sqrt[3]{\sqrt[3]{10}}} \ldots\]evaluates to a real number. What is that number?

\pagebreak

\subsection{2020 AMC 10A 21}

There exists a unique strictly increasing sequence of nonnegative integers $a_1 < a_2 < … < a_k$ such that\[\frac{2^{289}+1}{2^{17}+1} = 2^{a_1} + 2^{a_2} + … + 2^{a_k}.\]What is $k?$



\end{document}