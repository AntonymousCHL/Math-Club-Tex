\documentclass{article}

\usepackage[utf8]{inputenc}
\usepackage[letterpaper, total={6in, 9in}]{geometry}
\usepackage{amsmath}
\usepackage{natbib}
\usepackage{wrapfig}
\usepackage{graphicx}
\usepackage{amssymb}
\usepackage{tikz}


\title{Algebra 1 - Set Theory and Logic}
\author{TSS Math Club}
\date{Dec 2022}

\begin{document}
\large

\maketitle

\section{Four Fundamental Proof Techniques:}

\subsection{Direct Proof (Proof by Construction)}
In a constructive proof one attempts to demonstrate $P \implies Q$ directly. This is
the simplest and easiest method of proof available to us. There are only two
steps to a direct proof (the second step is, of course, the tricky part):
\begin{itemize}
    \item Assume that P is true.
    \item Use P to show that Q must be true.
\end{itemize}
\subsubsection{Example}
Prove odd number + odd number = even number.
\vspace{40px}

\subsection{Proof by Contradiction}
The proof by contradiction is grounded in the fact that any proposition must
be either true or false, but not both true and false at the same time. The method of proof by contradiction:
\begin{itemize}
    \item Assume that P is true.
    \item Assume that $\neg$ Q is true
    \item Use P and $\neg$ Q to demonstrate a contradiction.
\end{itemize}
\subsubsection{Example}
Prove $\sqrt{2}$ is irrational number.
\pagebreak
\subsection{Proof by Induction}
Proof by induction is a very powerful method in which we use recursion to
demonstrate an infinite number of facts in a finite amount of space. In order to prove by induction, you need to:
\begin{itemize}
    \item Show that a propositional form P(x) is true for some basis case.
    \item Assume that P(n) is true for some n, and show that this implies that
    P(n + 1) is true.
    \item  Then, by the principle of induction, the propositional form P(x) is true
    for all n greater or equal to the basis case.
\end{itemize}
\subsubsection{Example}
Prove $\sum_{i=1}^n i =\frac{n(n+1)}{2}$
\vspace{60px}
\subsection{Proof by Contrapositive}
From first-order logic we know that the implication $P \implies Q$ is equivalent
to $\neg P \implies \neg Q$. The second proposition is called the contrapositive of the first
proposition. By saying that the two propositions are equivalent we mean that
if one can prove $P \implies Q$ then they have also proved $\neg P \implies \neg Q$, and vice versa.
\subsubsection{Example}
Prove $n^2$ even implies $n$ even.
\pagebreak

\section{Set Theory}
\subsection{Definition of a Set:}
\vspace{20px}
\subsection{Construct a Set}
\subsubsection{Construct a Set from a List of Ojbects:}
\vspace{30px}
\subsubsection{Construct a Set from a Given Set by Setting Restrictions:}
\vspace{30px}

\subsection{Basic Sets:}
\vspace{50px}

\subsection{Relations:}
\subsubsection{Definition: $x \in A$}
\vspace{30px}
\subsubsection{Definition: $A \subset B$}
\vspace{30px}
\subsubsection{Definition: $A=B$}
\vspace{30px}
\subsubsection{Lemma: $A=B \iff A \subset B \land B \subset A$}
\pagebreak

\subsection{Operations:}
\subsubsection{Definition: $A-B$}
\vspace{30px}
\subsubsection{Definition: $A \cup B$}
\vspace{30px}
\subsubsection{Definition: $A \cap B$}
\vspace{30px}
\subsection{Ordered Pair:}

\subsubsection{Definition:}
\vspace{20px}

\subsubsection{Property:}
\vspace{30px}

\subsection{Cartesian Product}

\subsubsection{Definition:}
\vspace{30px}

\pagebreak
\section{Relation and Function}
\subsection{Relation}
\subsubsection{Definition:}
\vspace{30px}
\subsection{Function}
\subsubsection{Definition:}
\begin{itemize}
    \item Function
    \item Domain
    \item Codomain
    \item Range
\end{itemize}
\subsubsection{Injection (one-to-one):}
\vspace{40px}
\subsubsection{Surjection (onto):}
\vspace{40px}
\subsubsection{Bijection (one-to-one and onto):}
\vspace{40px}
\subsubsection{Inverse Function:}
\pagebreak
\section{Cardinality}
\subsection{Index Set $J_n$}
\vspace{30px}
\subsection{Cardinlity of Finite Set}
\vspace{30px}
\subsection{Infinite Countable Set}
\vspace{30px}
\subsection{Infinite Uncountable Set}
\vspace{30px}
\subsection{Two Sets with the Same Cardinality}
\end{document}

