\documentclass{article}

\usepackage[utf8]{inputenc}
\usepackage[letterpaper, total={6in, 9in}]{geometry}
\usepackage{amsmath}
\usepackage{natbib}
\usepackage{wrapfig}
\usepackage{graphicx}
\usepackage{amssymb}
\usepackage{tikz}
\usepackage{array}

\title{Number Theory}
\author{TSS Math Club}
\date{March 2023}

\begin{document}
\large

\maketitle

\section{Integers}
\subsection{Division with Remainder}
\subsubsection{Example}
Find the quotient and remainder when 102 is divided 5.
\vspace{20px}
\subsubsection{Example}
Find the quotient and remainder when 213 is divided 7.
\vspace{20px}
\subsection{Divisibility}
\subsubsection{Definition}
\vspace{20px}
\subsubsection{Notation}
$a|b$
\subsubsection{Theorems}
\begin{itemize}
    \item $a|b \text{ and } b|c \implies a|c$
    \item $a|b \implies a|cb$
    \item $a|b \text{ and } a|c \implies a|mb+nc$
\end{itemize}

\subsection{GCD and LCM}
\subsubsection{Defnition}
\begin{itemize}
    \item GCD:
    \item LCM:
\end{itemize}
\subsubsection{Notations}
\begin{itemize}
    \item GCD:
    \item LCM:
\end{itemize}

\subsubsection{Example}
\begin{itemize}
    \item (10,5)=\hspace{74px}[10,5]=
    \item (3,2)=\hspace{80px}[3,2]=
    \item (0,n)=\hspace{80px}[0,n]=
    \item (n,1)=\hspace{80px}[n,1]=
\end{itemize}

\subsubsection{Theorem}

If $(a,b)=d$, then $(a/d,b/d)=1$\\
Proof:
\vspace{40px}

\subsubsection{Theorem}

If $a=bq+r$, then $(a,b)=(b,r)$\\
Proof:
\vspace{40px}

\subsubsection{Euclidean Algorithm}
\pagebreak

\subsubsection{Theorem}
If$(a,b)=d$, then exist integers $x, y$ such that
$$ax+by=d$$
Proof:
\vspace{60px}

\subsubsection{Corollary}
If $d|ab$ and $(d,a)=1$, then $d|b$\\
Proof:
\vspace{40px}

\subsection{Primes and UFD}
\subsubsection{Primes}
Definition:
\vspace{20px}
\subsubsection{Lemma}
If $n$ is composite, the there is a divider $d$ such that $d \le n^{\frac{1}{2}}$\\
Proof:
\vspace{20px}
\subsubsection{Lemma}
If $n$ is composite, the there is a prime divider $p$ such that $p \le n^{\frac{1}{2}}$
\subsubsection{Euclid's Lemma}
If $p$ is a prime and $p|ab$ then $p|a$ or $p|b$.\\
Proof:

\pagebreak
\subsubsection{Extended Euclid's Lemma 1}
If $p$ is a prime and $p|a_1 a_2 ... a_n $ then $p|a_i$.

\subsubsection{Extended Euclid's Lemma 2}
If $p$ and $q_i$ are primes and $p|q_1 q_2 ... q_n $ then $p=q_i$.

\subsubsection{$\mathbb{Z}$ is UFD (Unique Factorization Domain)}
Any positive integer can be written as a product of primes in one and only one way.\\
Proof:
\vspace{50px}

\subsubsection{GCD and LCM in Terms of Factorization}
\vspace{50px}

\subsubsection{Theorem $(a,b)[a,b]=ab$}

\vspace{50px}
\section{Diophantine Equations}
\subsection{Definition}
\vspace{20px}
\subsection{Use Divisibility}
\subsubsection{Example}
Given $x,y$ are integers and $xy=30$, find ordered pair $(x,y)$.
\pagebreak
\subsubsection{Example}
Given $x,y$ are integers and
$$y=\frac{x^3 + 7 x - 10}{x+3},$$
find ordered pair $(x,y)$.

\vspace{60px}

\subsubsection{Simon's Favourite Factoring Trick}
Given $x,y$ are integers and
$$3x+xy+3y+31=0,$$
find ordered pair $(x,y)$.
\vspace{60px}

\subsection{Solve Linear Diophantine Equations}
\subsubsection{Definition}

Solve $ax+by=c$ for integers $x,y$.

\subsubsection{Theorem}

For the equation above, if $(a,b)|c$, then there are infinite number of solutions. If $(a,b) \nmid c$, then there is no solution.

\subsubsection{Example}
Solve $3x+4y=10$.
\vspace{50px}

\subsubsection{Example}
Solve $8x+4y=6$.


\subsubsection{Example}
Solve $6x+9y=24$.
\vspace{50px}

\section{Congruences and Modulo}

\end{document}