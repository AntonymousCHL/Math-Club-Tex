\documentclass{article}

\usepackage[utf8]{inputenc}
\usepackage[letterpaper, total={6in, 9in}]{geometry}
\usepackage{amsmath}
\usepackage{natbib}
\usepackage{wrapfig}
\usepackage{graphicx}
\usepackage{amssymb}
\usepackage{tikz}
\usepackage{array}

\title{Number Theory}
\author{TSS Math Club}
\date{March 2023}

\begin{document}
\large

\maketitle

\section{Integers}
\subsection{Division with Remainder}
\subsubsection{Example}
Find the quotient and remainder when 102 is divided 5.
\vspace{20px}
\subsubsection{Example}
Find the quotient and remainder when 213 is divided 7.
\vspace{20px}
\subsection{Divisibility}
\subsubsection{Definition}
\vspace{20px}
\subsubsection{Notation}
$a|b$
\subsubsection{Theorems}
\begin{itemize}
    \item $a|b \text{ and } b|c \implies a|c$
    \item $a|b \implies a|cb$
    \item $a|b \text{ and } a|c \implies a|mb+nc$
\end{itemize}

\subsection{GCD and LCM}
\subsubsection{Defnition}
\begin{itemize}
    \item GCD:
    \item LCM:
\end{itemize}
\subsubsection{Notations}
\begin{itemize}
    \item GCD:
    \item LCM:
\end{itemize}

\subsubsection{Example}
\begin{itemize}
    \item (10,5)=\hspace{74px}[10,5]=
    \item (3,2)=\hspace{80px}[3,2]=
    \item (0,n)=\hspace{80px}[0,n]=
    \item (n,1)=\hspace{80px}[n,1]=
\end{itemize}

\subsubsection{Theorem}

If $(a,b)=d$, then $(a/d,b/d)=1$\\
Proof:
\vspace{40px}

\subsubsection{Theorem}

If $a=bq+r$, then $(a,b)=(b,r)$\\
Proof:
\vspace{40px}

\subsubsection{Euclidean Algorithm}
\pagebreak

\subsubsection{Theorem}
If$(a,b)=d$, then exist integers $x, y$ such that
$$ax+by=d$$
Proof:
\vspace{60px}

\subsubsection{Corollary}
If $d|ab$ and $(d,a)=1$, then $d|b$\\
Proof:
\vspace{40px}

\subsection{Primes and UFD}
\subsubsection{Primes}
Definition:
\vspace{20px}
\subsubsection{Lemma}
If $n$ is composite, the there is a divider $d$ such that $d \le n^{\frac{1}{2}}$\\
Proof:
\vspace{20px}
\subsubsection{Lemma}
If $n$ is composite, the there is a prime divider $p$ such that $p \le n^{\frac{1}{2}}$
\subsubsection{Euclid's Lemma}
If $p$ is a prime and $p|ab$ then $p|a$ or $p|b$.\\
Proof:

\pagebreak
\subsubsection{Extended Euclid's Lemma 1}
If $p$ is a prime and $p|a_1 a_2 ... a_n $ then $p|a_i$.

\subsubsection{Extended Euclid's Lemma 2}
If $p$ and $q_i$ are primes and $p|q_1 q_2 ... q_n $ then $p=q_i$.

\subsubsection{$\mathbb{Z}$ is UFD (Unique Factorization Domain)}
Any positive integer can be written as a product of primes in one and only one way.\\
Proof:
\vspace{50px}

\subsubsection{GCD and LCM in Terms of Factorization}
\vspace{50px}

\subsubsection{Theorem}
$(a,b)[a,b]=ab$
\vspace{50px}
\subsubsection{Theorem}
Number of divisor $d(n)=$


\vspace{50px}

\section{Diophantine Equations}
\subsection{Definition}
\vspace{20px}
\subsection{Use Divisibility}
\subsubsection{Example}
Given $x,y$ are integers and $xy=30$, find ordered pair $(x,y)$.
\pagebreak
\subsubsection{Example}
Given $x,y$ are integers and
$$y=\frac{x^3 + 7 x - 10}{x+3},$$
find ordered pair $(x,y)$.

\vspace{60px}

\subsubsection{Simon's Favourite Factoring Trick}
Given $x,y$ are integers and
$$3x+xy+3y+31=0,$$
find ordered pair $(x,y)$.
\vspace{60px}

\subsection{Solve Linear Diophantine Equations}
\subsubsection{Definition}

Solve $ax+by=c$ for integers $x,y$.

\subsubsection{Theorem}

For the equation above, if $(a,b)|c$, then there are infinite number of solutions. If $(a,b) \nmid c$, then there is no solution.

\subsubsection{Example}
Solve $3x+4y=10$.
\vspace{50px}

\subsubsection{Example}
Solve $8x+4y=6$.


\subsubsection{Example}
Solve $6x+9y=24$.
\vspace{50px}

\section{Congruences and Modulo}
\subsection{Definition}
If $a$ is congruent to $b$ modulo $m$ ($a \equiv b$ $(m)$) or ($a \equiv b$ (mod $m$)), then $m|a-b$.

\subsection{Congruences and Remainder}
\subsubsection{Theorem}
Every integer is congruent $m$ to exactly one of $0, 1, ... , m-1$.
\vspace{40px}

\subsubsection{Theorem}
$a \equiv b$ ($m$) iff $a$ and $b$ leave the same remainder on division by $m$.
\vspace{40px}

\subsection{Operations under modulo}
\subsubsection{Lemma}
\begin{itemize}
    \item $a \equiv a \ (m)$.
    \item If $a \equiv b \ (m)$, then $b \equiv a \ (m)$.
    \item If $a \equiv b \ (m)$ and $c \equiv d \ (m)$, then $a+b \equiv c+d \ (m)$.
    \item If $a \equiv b \ (m)$ and $c \equiv d \ (m)$, then $ab \equiv cd \ (m)$.
\end{itemize}
\vspace{60px}
\subsubsection{Theorem}
If $ac \equiv bc \ (m)$ and $(c,m)=1$, then $a \equiv b \ (m)$
\vspace{40px}
\subsubsection{Theorem}
If $ac \equiv bc \ (m)$ and $(c,m)=d$, then $a \equiv b \ (m/d)$
\vspace{40px}
\subsection{Problems}
\subsubsection{Problem}
Find the least residue of 1492 (mod 4), (mod 10), (mod 101).
\vspace{40px}
\subsubsection{Problem}
Solve $2x \equiv 4 \ (6)$.
\vspace{40px}
\subsubsection{Problem}
Prove $m^2 \equiv \text{ 0 or 1 } (4)$
\vspace{40px}
\subsubsection{Problem}
Solve $m^2+n^2=1023$
\vspace{40px}
\subsubsection{Problem}
Show every integer is congruent to (mod 9) to the sum of its digits.
\pagebreak

\section{Linear Congruences}
We will try to solve the linear equation $ax \equiv b \pmod{m}$ in this section.
\subsection{General Theory}
\subsubsection{Theorem}
If $(a, m) \nmid b$, then $ax \equiv b \ (m)$ has no solutions.
\vspace{50px}
\subsubsection{Theorem}
If $(a, m) \nmid 1$, then $ax \equiv b \ (m)$ has exactly one solution mod $m$.
\vspace{50px}
\subsubsection{Theorem}
If $(a, m) \nmid d$, then $ax \equiv b \ (m)$ has exactly one solution mod $m/d$.
\vspace{50px}
\subsection{Problems}
\subsubsection{Problem}
Solve $2x\equiv 1 \ (17)$
\vspace{40px}
\subsubsection{Problem}
Solve $3x\equiv 1 \ (17)$
\vspace{40px}
\subsubsection{Problem}
Solve $15x+16y=17$
\vspace{40px}
\subsection{Chinese Remainder Theorem (CRT)}
If the $n_{i}$ are pairwise coprime, and if $a_1, ..., a_k$ are any integers, then the system
$$\begin{aligned}
    x & \equiv a_{1} \quad\left(\bmod n_{1}\right) \\
    & \vdots \\
    x & \equiv a_{k} \quad\left(\bmod n_{k}\right)
    \end{aligned}$$
has one solution mod $N=n_1 n_2 ... n_k$.
\subsubsection{Example}
Solve:
\[x\equiv 1 \pmod 2\]\[4x\equiv 3 \pmod 5\]
\vspace{50px}
\subsubsection{Problem}
Find the remainder when divided by 10 of the following:
$$4^{4^{4^{4^{4^{4^{4{...^4}}}}}}}$$
(There are 2023 4's in total).
\vspace{50px}

\section{Wilson's, Fermat's, Euler's Theorems}
\subsection{Wilson's Theorem}
A natural number $n > 1$ is a prime number if and only if the product of all the
positive integers less than n is one less than a multiple of $n$ or 
$$(n-1) ! \equiv-1 \quad(\bmod n).$$

\subsection{Fermat's Little Theorem}
If $a$ is not divisible by the prime $p$, then 
$$a^{p-1} \equiv 1 \pmod{p}$$

\subsubsection{Example}
What is the least residue of $1945^8 \pmod{7}$
\vspace{30px}
\subsubsection{Example}
What is the least residue of $2025^22 \pmod{11}$
\vspace{30px}


\subsection{Euler's Theorem}
\subsubsection{Euler's Totient Function}
Euler's totient function $ \varphi(n)$ counts the positive integers up to a given integer $n$ that are relatively prime to $n$.

\subsubsection{Example}
Find $ \varphi(24)$
\vspace{30px}
\subsubsection{Euler's Totient Function is Multiplicative}
If $(a,b)=1$, then $\varphi(ab)=\varphi(a)\varphi(b)$.

\subsubsection{Example}
Find $ \varphi(2)$, $ \varphi(5)$, $ \varphi(10)$.
\vspace{40px}
\subsubsection{Euler's Totient Function for $p^n$}
$ \varphi(p^n)=p^n-p^{n-1}$
\vspace{30px}
\subsubsection{Euler's Totient Function General Formula}
\vspace{50px}
\subsubsection{Euler's Theorem}
If $(a,m)=1$, then 
$$a^{\varphi(m)} \equiv 1 \pmod{m}$$
\subsubsection{Example}
What is the least residue of $2023^{41} \pmod{100}$

\end{document}